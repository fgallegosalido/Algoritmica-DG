\documentclass[11pt,spanish]{article} % Idioma
\usepackage{babel}
\usepackage[T1]{fontenc}
\usepackage{textcomp}
\usepackage[utf8]{inputenc} % Puede depender del instrucción, sistema o editor
\usepackage{wrapfig} % Imagenes 
% \graphicspath{ {./imagenes/} }

%\usepackage[left=2cm,top=2.5cm,right=2cm,bottom=2.5cm]{geometry} % Márgenes
%\usepackage{pstricks} % Gráficas, movilidad, árboles y otros

\usepackage{amssymb, amsmath} % Símbolos matemáticos
\usepackage{amsthm} % Teoremas, lemas, pruebas...
\usepackage{cancel} % Cancelar expresiones
\usepackage{multirow} % Tablas
\usepackage{graphicx} % Inserción de imágenes
\usepackage{xcolor} % Colores
\usepackage{color}
\definecolor{gray97}{gray}{.97}
\definecolor{gray75}{gray}{.75}
\definecolor{gray45}{gray}{.45}

\usepackage[hidelinks]{hyperref}  % Enlaces
\usepackage{multirow} % Tablas

\usepackage{listings} % Escribir código en diferentes lenguajes de programación
\lstset{ frame=Ltb,
framerule=0pt,
aboveskip=0.5cm,
framextopmargin=3pt,
framexbottommargin=3pt,
framexleftmargin=0.4cm,
framesep=0pt,
rulesep=.4pt,
backgroundcolor=\color{gray97},
rulesepcolor=\color{black},
%
stringstyle=\ttfamily,
showstringspaces = false,
basicstyle=\small\ttfamily,
commentstyle=\color{gray45},
keywordstyle=\bfseries,
%
numbers=left,
numbersep=15pt,
numberstyle=\tiny,
numberfirstline = false,
breaklines=true,
}

\title{Memoria de algor\'itmica}
\author{Rubén Morales Pérez 
		\and Francisco Javier Morales Piqueras
		\and Bruno Santindrian Manzanedo 
		\and Ignacio de Loyola Barragan Lozano
		\and Francisco Leopoldo Gallego Salido}
\date{\today}


% % % % % % % % % % % % % % % % % % % % % % % % % % % % % % % % % 
%					 Inicio del documento
% % % % % % % % % % % % % % % % % % % % % % % % % % % % % % % % %
\begin{document} 
\maketitle
\tableofcontents % Generando el indice
\newpage
\setlength\parindent{0pt} % Quitamos la sangría

%%%%%%%%%%%%%%%%%%%%%%%%%%%%%%%%%%%%%%%%%%%%%%%%%%%%%%%%%%%%%%%%%%%%%%%%%%%%%%%%%%


\section{Explicaci\'on del m\'etodo utilizado}
Para la obtenci\'on de los datos deseados hemos realizado un script de bash que genera las tablas de datos y las gráficas con su correspondiente ajuste. 
\begin{lstlisting}[language=bash]
#!/bin/bash

if [ $# -ne 1 ]
then
    echo "Uso: $0 <nombre>"
    exit 1
fi

# HEAPSORT
g++ -std=c++11 ../src/heapsort.cpp
nelementos=200
echo "" > datos.dat
while [ $nelementos -lt 10000 ]; do
    ./a.out $nelementos >> datos.dat
    let nelementos=nelementos+100
done

gnuplot ./gnuplot/heapsort.gp # Salida: "fichero.jpeg"

mkdir ../Graficas/Heapsort 2> /dev/null
mkdir ../Graficas/Heapsort/Datos 2> /dev/null
mv fichero.jpeg ../Graficas/Heapsort/heapsortO0_$1.jpeg
mv datos.dat ../Graficas/Heapsort/Datos/heapsortO0_$1.dat
echo "Heapsort completado"


# MERGESORT
g++ -std=c++11 ../src/mergesort.cpp
nelementos=200
echo "" > datos.dat
while [ $nelementos -lt 10000 ]; do
    ./a.out $nelementos >> datos.dat
    let nelementos=nelementos+100
done

gnuplot ./gnuplot/mergesort.gp # Salida: "fichero.jpeg"

mkdir ../Graficas/Mergesort 2> /dev/null
mkdir ../Graficas/Mergesort/Datos 2> /dev/null
mv fichero.jpeg ../Graficas/Mergesort/mergesortO0_$1.jpeg
mv datos.dat ../Graficas/Mergesort/Datos/mergesortO0_$1.dat
echo "Mergesort completado"


# INSERCION
g++ -std=c++11 ../src/insercion.cpp
nelementos=200
echo "" > datos.dat
while [ $nelementos -lt 10000 ]; do
    ./a.out $nelementos >> datos.dat
    let nelementos=nelementos+100
done

gnuplot ./gnuplot/insercion.gp # Salida: "fichero.jpeg"

mkdir ../Graficas/Insercion 2> /dev/null
mkdir ../Graficas/Insercion/Datos 2> /dev/null
mv fichero.jpeg ../Graficas/Insercion/insercionO0_$1.jpeg
mv datos.dat ../Graficas/Insercion/Datos/insercionO0_$1.dat
echo "Insercion completado"


# SELECCION
g++ -std=c++11 ../src/seleccion.cpp
nelementos=200
echo "" > datos.dat
while [ $nelementos -lt 10000 ]; do
    ./a.out $nelementos >> datos.dat
    let nelementos=nelementos+100
done

gnuplot ./gnuplot/insercion.gp # Salida: "fichero.jpeg"

mkdir ../Graficas/Seleccion 2> /dev/null
mkdir ../Graficas/Seleccion/Datos 2> /dev/null
mv fichero.jpeg ../Graficas/Seleccion/seleccionO0_$1.jpeg
mv datos.dat ../Graficas/Seleccion/Datos/seleccionO0_$1.dat
echo "Seleccion completado"


# QUICKSORT
g++ -std=c++11 ../src/quicksort.cpp
nelementos=200
echo "" > datos.dat
while [ $nelementos -lt 10000 ]; do
    ./a.out $nelementos >> datos.dat
    let nelementos=nelementos+100
done

gnuplot ./gnuplot/quicksort.gp # Salida: "fichero.jpeg"

mkdir ../Graficas/Quicksort 2> /dev/null
mkdir ../Graficas/Quicksort/Datos 2> /dev/null
mv fichero.jpeg ../Graficas/Quicksort/quicksortO0_$1.jpeg
mv datos.dat ../Graficas/Quicksort/Datos/quicksortO0_$1.dat
echo "Quicksort completado"


# BURBUJA
g++ -std=c++11 ../src/burbuja.cpp
nelementos=200
echo "" > datos.dat
while [ $nelementos -lt 10000 ]; do
    ./a.out $nelementos >> datos.dat
    let nelementos=nelementos+100
done

gnuplot ./gnuplot/burbuja.gp # Salida: "fichero.jpeg"

mkdir ../Graficas/Burbuja 2> /dev/null
mkdir ../Graficas/Burbuja/Datos 2> /dev/null
mv fichero.jpeg ../Graficas/Burbuja/burbujaO0_$1.jpeg
mv datos.dat ../Graficas/Burbuja/Datos/burbujaO0_$1.dat
echo "Burbuja completado"


# FIBONACCI
g++ -std=c++11 ../src/fibonacci.cpp
nelementos=1
echo "" > datos.dat
while [ $nelementos -lt 50 ]; do
    ./a.out $nelementos >> datos.dat
    let nelementos=nelementos+2
done

gnuplot ./gnuplot/fibonacci.gp # Salida: "fichero.jpeg"

mkdir ../Graficas/Fibonacci 2> /dev/null
mkdir ../Graficas/Fibonacci/Datos 2> /dev/null
mv fichero.jpeg ../Graficas/Fibonacci/fibonacciO0_$1.jpeg
mv datos.dat ../Graficas/Fibonacci/Datos/fibonacciO0_$1.dat
echo "Fibonacci completado"


# HANOI
g++ -std=c++11 ../src/hanoi.cpp
nelementos=3
echo "" > datos.dat
while [ $nelementos -lt 30 ]; do
    ./a.out $nelementos >> datos.dat
    let nelementos=nelementos+1
done

gnuplot ./gnuplot/hanoi.gp # Salida: "fichero.jpeg"

mkdir ../Graficas/Hanoi 2> /dev/null
mkdir ../Graficas/Hanoi/Datos 2> /dev/null
mv fichero.jpeg ../Graficas/Hanoi/hanoiO0_$1.jpeg
mv datos.dat ../Graficas/Hanoi/Datos/hanoiO0_$1.dat
echo "Hanoi completado"


# FLOYD
g++ -std=c++11 ../src/floyd.cpp
nelementos=200
echo "" > datos.dat
while [ $nelementos -lt 1000 ]; do
    ./a.out $nelementos >> datos.dat
    let nelementos=nelementos+10
done

gnuplot ./gnuplot/floyd.gp # Salida: "fichero.jpeg"

mkdir ../Graficas/Floyd 2> /dev/null
mkdir ../Graficas/Floyd/Datos 2> /dev/null
mv fichero.jpeg ../Graficas/Floyd/floydO0_$1.jpeg
mv datos.dat ../Graficas/Floyd/Datos/floydO0_$1.dat
echo "Floyd completado"

rm a.out
rm fit.log
\end{lstlisting}

Para la obtención de las gráficas de forma directa utilizamos script de gnuplot que tienen la forma siguiente, en este caso adjuntamos "burbuja.gp". 

\begin{lstlisting}[language=gnuplot]
set terminal jpeg
set output "fichero.jpeg"

set title "Eficiencia burbuja"
set xlabel "Tamano del vector"
set ylabel "Tiempo (s)"
set fit quiet
f(x) = a*x*x+b*x+c
fit f(x) "datos.dat" via a, b, c
plot "datos.dat", f(x)
\end{lstlisting}

Los diferentes ajustes se han conseguido así:

\begin{lstlisting}[language=gnuplot]
f(x) = a*x*x*x+b*x*x+c*x+d
g(x) = a*x*x+b*x+c
h(x) = a*x*(log(x)/log(2))
i(x) = a*(((1+sqrt(5))/2)**x)
\end{lstlisting}


\newpage
%%%%%%%%%%%%%%%%%%%%%%%%%%%%%%%%%%%%%%%%%%%%%%%%%%%%%%%%%%%%%%%%%%%%%%%%%%%%%%%%%%

\section{C\'alculo de la eficiencia emp\'irica}
%\hspace*{1cm}\textbf{Ejercicio 1.}

\subsection{Tabla con los algor\'itmos cuadr\'aticos}
\subsection{Tabla con los algor\'itmos c\'ubicos}
\subsection{Tabla con los algor\'itmos nlog(n)}
\subsection{Tabla con el algor\'itmo de Fibonacci}
\subsection{Tabla con el algor\'itmo de Hanoi}
\subsection{Tabla con los algoritmos de ordenaci\'on}

\newpage
%%%%%%%%%%%%%%%%%%%%%%%%%%%%%%%%%%%%%%%%%%%%%%%%%%%%%%%%%%%%%%%%%%%%%%%%%%%%%%%%%%

\section{Gr\'aficas}

\subsection{Ordenaci\'ion}
\subsubsection{Burbuja}
\subsubsection{Inserci\'on}
\subsubsection{Selecci\'on}
\subsubsection{Mergesort}
\subsubsection{Quicksort}
\subsubsection{Heapsort}
\subsubsection{Comparativa algoritmos de ordenaci\'ion}


\subsection{Fibonacci}
\subsection{Hanoi}
\subsection{Floyd}

\newpage
%%%%%%%%%%%%%%%%%%%%%%%%%%%%%%%%%%%%%%%%%%%%%%%%%%%%%%%%%%%%%%%%%%%%%%%%%%%%%%%%%%

% % % % % % % % % % % % % % % % % % % % % % % % % % % % % % % % % 
%					 Bibliografía
% % % % % % % % % % % % % % % % % % % % % % % % % % % % % % % % %
% input{bibliografia}

\end{document}
