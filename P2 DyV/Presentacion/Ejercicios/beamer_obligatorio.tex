
\section{Mezclando k vectores ordenados}

\subsection{Automatización}
\begin{frame}{Scripts}
	\begin{block}{Script}
		Podemos obtener los datos fijando el n\'umero de vectores usados.
	\end{block}
	
	\begin{exampleblock}{script.sh}
	g++ -std=c++11 ../src/mezcla.cpp

	nelementos=10

	while [ \$nelementos -lt 2500 ]; do
    
    		./a.out \$nelementos 200 3
    
    		let nelementos=nelementos+25
	
	done
	\end{exampleblock}
\end{frame}

%%%%%%%%%%%%%%%%%%%%%%%%%%%%%%%%%%%%%%%%%%%%

\begin{frame}
	\begin{block}{Script}
		Si queremos fijar el n\'umero de vectores usaremos
	\end{block}
	
	\begin{exampleblock}{script.sh}
	kvectores=10

	while [ \$kvectores -lt 2500 ]; do
    
    		./a.out 200 \$kvectores 2
    	
    		let kvectores=kvectores+25

	done
	\end{exampleblock}
\end{frame}

%%%%%%%%%%%%%%%%%%%%%%%%%%%%%%%%%%%%%%%%%%%%

\begin{frame}
	\begin{block}{Script}
	Datos en 3 dimensiones, número de vectores, elementos del vector, y tiempo del algoritmo
	\end{block}
	
	\begin{exampleblock}{script.sh}
	nelementos=10

	nvectores=10

	while [ \$nelementos -lt 1000 ]; do
   	
   		./a.out \$nelementos 10 1
      		
   		...
   		
   		./a.out \$nelementos 910 1

   		let nelementos=nelementos+100
	done
	\end{exampleblock}
\end{frame}

%%%%%%%%%%%%%%%%%%%%%%%%%%%%%%%%%%%%%%%%%%%%

\begin{frame}{Scripts de gnuplot}
	\begin{block}{Gnuplot}		
		Los datos están en "datos.dat". Ejecutamos		
		\hspace{1cm}\$ gnuplot algoritmo.gp
	\end{block}
	\pause
	
	\begin{columns}

	\begin{column}{5cm}
	\begin{exampleblock}{algoritmo.gp}
	set terminal pngcairo
	
	set output "grafica.png"

	set title "..."

	set xlabel "Vectores/Elementos del vector"

	set ylabel "Tiempo (s)"

	set fit quiet

	f(x) = ...

	fit f(x) "datos.dat" via a

	plot "datos.dat", f(x)
	\end{exampleblock}
	\end{column}
	\pause
	
	\begin{column}{5cm}
	\begin{block}{Funciones ajustadas}
		\[f(x) = a*x\]
		\[g(x) = a*x*x\]
		\[h(x) = a*x*(log(x)/log(2))\]
	\end{block}
	\end{column}
	
	\end{columns}
\end{frame}

%%%%%%%%%%%%%%%%%%%%%%%%%%%%%%%%%%%%%%%
\subsection{Ordenador usado}
\begin{frame}{Ordenador usado}
	\begin{alertblock}{Ordenador usado para la ejecuci\'on}
	HP Pavilion g series (Pavilion g6)

	Sistema operativo: ubuntu 14.04 LTS

	Memoria: 3.8 GiB (4Gb)

	Procesador: Inter Core i3-2330M CPU @ 2.20GHz x 4

	Gráficos: Intel Sandybridge Mobile

	Tipo de SO: 64 bits

	Disco: 487.9 GB	
	\end{alertblock}
\end{frame}


%%%%%%%%%%%%%%%%%%%%%%%%%%%%%%%%%%%%%%

\begin{frame}{Problema}
	\begin{block}{Mezclando k vectores ordenados}
	Se tienen $k$ vectores ordenados (de menor a mayor), cada uno con $n$ elementos, y 				queremos combinarlos en un único vector ordenado (con $kn$ elementos)
	\end{block}

	\begin{block}{Cota superior}
	Es posible imponer una cota superior teórica. Teniendo en cuenta que hay kn elementos, 	si aplicásemos un algoritmo 	de ordenación con eficiencia $O(n) = nlog(n)$ deducimos 			que podemos encontrar un algoritmo de ordenación básica con eficiencia $O(k, n) = 
	nklog(nk)$. Tomar los $k$ vectores como uno solo no aprovecha aún el hecho de que 			partes del vector están ordenadas.
	\end{block}
\end{frame}

%%%%%%%%%%%%%%%%%%%%%%%%%%%%%%%%%%%%%%%
\subsection{Fuerza bruta}
\begin{frame}{Fuerza bruta}
	\begin{block}{Algoritmo}
	En cada paso elegimos el mínimo de los primeros elementos de los $k$ vectores, será el 	primer elemento del vector creciente resultante.
	Para el siguiente paso descartamos ese elemento y calculamos otra vez el mínimo, lo 			insertamos al final del vector resultante y así sucesivamente.

	Buscar el mínimo es $O(k)=k$ ya que el vector de índices tiene $k$ 							elementos, y lo repetimos $kn$ veces.
	\end{block}
	
	\begin{block}{Eficiencia}
	\[\sum_{i=1}^{kn}k = nk^2 \implies \ O(k,n)=nk^2\]
	\end{block}
\end{frame}

%%%%%%%%%%%%%%%%%%%%%%%%%%%%%%%%%%%%%%%%
\subsection{Divide y vencerás}
\begin{frame}{Divide y vencerás}
	\begin{block}{Algoritmo}
	Usaremos mergesort, pero con los primeros montículos ya creados, por tanto tendrá una 		constante oculta menor que usar mergesort para $kn$ datos arbitrarios. 
	
	En el proceso lo que haremos es ir mezclando las partes de dos en dos. El algoritmo 			que mezcla dos vectores en un único tiene eficiencia $O(n) = n$.
	\end{block}
\end{frame}

\begin{frame}
	\begin{block}{Código} % errorl
		\begin{figure}[h]
    		\centering
    		%\includegraphics[width=0.9\textwidth]{../Imagenes/dyv.png}
    		\label{fig:mesh1}
		\end{figure}
	\end{block}
\end{frame}

%%%%%%%%%%%%%%%%%%%%%%%%%%%%%%%%%%%%%

\begin{frame}
	\begin{block}{Eficiencia}
	Donde $k$ es el n\'umero de vectores y $n$ el n\'umero de elementos de cada vector:s

	\[T(k,n) = \left \{ 
	\begin{matrix} 
		2n & 				\mbox{si } k=2
	\\ 2T(k/2,n) + kn & 		\mbox{si } k>2
	\end{matrix}
	\right.\]
	\end{block}
\end{frame}

\begin{frame}
	\begin{block}{Desarrollo}
	Sustituyendo $k=2^m \implies$ $T(2^m, n) = 2T(2^{m-1}, n) + 2^mn$
	\[T(2^m, n) = 2\left[ T(2^{m-2}, n) + 2^{m-1}n \right] + 2^mn\]
	\begin{center}
	Para el caso gen\'erico, con $j \in \left[0,m-1\right] \cap\mathbb{N}$ y 						desarrollando:
	\end{center}
	\[T(2^m, n)	= 2^jT(2^{m-j}, n) + \sum_{i=1}^{m-1} 2^mn\]
	\[T(2^m, n) = 2^{m-1} T(2, n) + \sum_{i=1}^{m-1} 2^mn\]
	\[T(2^m, n) = 2^mn + (m-1) 2^mn = 2^mn[1+(m-1)] = 2^mnm\]
	\end{block}
\end{frame}

\begin{frame}{Eficiencia final}
	\begin{block}{Solución}
	Deshacemos el cambio de variable, $k=2^m \implies log_2(k)=m$:
	\[T(k,n) = knlog_2k\]
	\end{block}
\end{frame}

\subsection{Estudio emp\'irico e h\'ibrido fuerza bruta}
\begin{frame}{Ajuste fuerza bruta}
	\begin{block}
	Vamos a variar el n\'umero de vectores, la funci\'on que debemos ajustar es 
	$f(x) = ax^2$
	$\centering$
	
	%\fcolorbox{gray75}{gray97}{
		$a               = 1.77962\cdot 10^{-6}$
	%}
	\end{block}
	
	\begin{block}
	Para calcular los coeficientes de correlaci\'on hemos usado la función stats de gnuplot:
		%stats fuerza_bruta_kvectores\.dat using 2:(f(\$1))

	\end{block}
\end{frame}


\begin{frame}{Variando vectores}
	\begin{exampleblock}{Imagen}
	\begin{figure}[htb] 
	\centering
	\includegraphics[width=0.6\textwidth]														{../Obligatorio/Graficas/fuerza_bruta_kvectores.png}
	\caption{Fuerza bruta con 200 elementos cada vector} 
	\label{fig:f_kvectores} 
	\end{figure}
	\end{exampleblock}
\end{frame}

\begin{frame}
	\begin{block}
	Para la parte en la que cambiamos el n\'umero de elementos ajustamos la funci\'on 
	$f(x) = ax$ ya 	que en $T(k, n) = nklog_2k, \ klog_2k$ es una constante, concretamente 	$200$.

	\begin{center}
	$a               = 0.00031202$

	Correlation:  $r = 0.9934$
	\end{center}
	\end{block}
\end{frame}

\begin{frame}{Imagen}
	\begin{exampleblock}
	
	\begin{figure}[h] 
	\includegraphics[width=0.6\textwidth]
	{../Obligatorio/Graficas/fuerza_bruta_nelementos.png}
	\caption{Fuerza bruta con 200 vectores} 
	\end{figure}
	
	\end{exampleblock}
\end{frame}

\subsection{Mezcla con divide y vencer\'as}
\begin{frame}[Divide y vencerás]
	\begin{block}

	La funci\'on ajustada ha sido $f(x) = ax(log(x)/log(2))$

	\begin{center}	
	$a               = 4.52594\cdot 10^{-6}$
	Correlation:  $r = 0.9863$
	\end{center}
	\end{block}
\end{frame}

\begin{frame}[Imagen]
	\begin{block}
	
	\begin{figure}[h] 
	\centering
	\includegraphics[width=0.6\textwidth]{../Obligatorio/Graficas/dyv_kvectores.png}
	\caption{Divide y vencerás con 200 elementos cada vector} 
	\end{figure}
	
	\end{block}
\end{frame}

\begin{frame}
	\begin{block}
	
	Si ahora fijamos $k=200$ y hacemos variable el n\'umero de elementos debemos ajustar 			la funci\'on $f(x) = ax(log(x)/log(2))$

	\begin{center}
	$a               = 3.03036\cdot 10^{-6}$
	Correlation:  $r = 0.9933$
	\end{center}
	\end{block}
\end{frame}

\begin{frame}{Imagen}
	\begin{block}
	
	\begin{figure}[h] 
	\centering
	\includegraphics[width=0.6\textwidth]{../Obligatorio/Graficas/dyv_nelementos.png}
	\caption{Divide y venceras con 200 vectores} 
	\label{fig:d_nelementos} 
	\end{figure}
	
	\end{block}

\end{frame}