
Muchos sitios web intentan comparar las preferencias de dos usuarios para realizar sugerencias a partir de las preferencias de usuarios con gustos similares a los nuestros. Dado un ranking de $n$ productos (p.ej. pel\'iculas) mediante el cual los usuarios indicamos nuestras preferencias, un algoritmo puede medir la similitud de nuestras preferencias contando el n\'umero de inversiones: dos productos $i$ y $j$ est\'an \"invertidos\" en las preferencias de $A$ y $B$ si el usuario $A$ prefiere el producto $i$ antes que el $j$, mientras que el usuario $B$ prefiere el producto $j$ antes que el $i$. Esto es, cuantas menos inversiones existan entre dos rankings, m\'as similares ser\'an las
preferencias de los usuarios representados por esos rankings.

Por simplicidad podemos suponer que los productos se pueden identificar mediante enteros
$1, \cdots, n$, y que uno de los rankings siempre es $1,\cdots, n$ (si no fuese as\'i bastar\'ia reenumerarlos) y el otro es $a_1, a_2, \cdots, a_n$, de forma que dos productos $i$ y $j$ est\'an invertidos si $i < j$ pero $a_i > a_j$.
De esta forma nuestra representaci\'on del problema ser\'a un vector de enteros $v$ de tamaño $n$, de forma que $v[i] = a_i\ / \ i = 1, \cdots, n$.

El objetivo es diseñar, analizar la eficiencia e implementar un algoritmo \"divide y vencer\'as\" para medir la similitud entre dos rankings. Compararlo con el algoritmo de \"fuerza bruta\" obvio. Realizar tambi\'en un estudio emp\'irico e h\'ibrido de la eficiencia de ambos algoritmos.

\subsection{Fuerza bruta}

\subsection{Divide y vencer\'as}