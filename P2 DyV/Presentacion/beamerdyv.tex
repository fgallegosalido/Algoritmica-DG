% Pablo Baeyens (@pbaeyens)
% Email: pbaeyens31+github@gmail.com
% Licencia: CC BY-SA 3.0

%% Paquetes y configuración %

% Beamer
\PassOptionsToPackage{unicode}{hyperref}  % Evita errores con caracteres no ASCII
\PassOptionsToPackage{naturalnames}{hyperref} % tex.stackexchange.com/questions/10555
\documentclass[compress]{beamer}

% Idioma
\usepackage[spanish]{babel} % Traducciones
\usepackage[utf8]{inputenc} % Uso de caracteres UTF-8
\usepackage{lmodern}        % Fuentes de tamaño arbitrario
\usepackage[T1]{fontenc}    % Permite copiar y evita errores
\uselanguage{Spanish}       % Traducciones beamer
\languagepath{Spanish}      % (tex.stackexchange.com/questions/168208)

% Matemáticas
\usepackage{amsfonts}
\usepackage{amsmath}
\usepackage{amssymb}

% Colores
\definecolor{backg}{HTML}{F2F2F2}    % Fondo
\definecolor{title}{HTML}{bdc3d1}    % Títulos
\definecolor{comments}{HTML}{BDBDBD} % Comentarios
\definecolor{keywords}{HTML}{08388c} % Palabras clave
\definecolor{strings}{HTML}{FA5858}  % Strings
\definecolor{links}{HTML}{2C2C95}    % Enlaces
\definecolor{bars}{HTML}{045FB4}     % Barras (gráfico)

% Código
\usepackage{listings}
\lstset{
language=[LaTeX]TeX,
basicstyle=\footnotesize,
morekeywords={href,uselanguage,languagepath,column},
otherkeywords={pause,usetheme,usecolortheme,useinnertheme,titlepage,tableofcontents,subtitle},
breaklines=true,
backgroundcolor=\color{backg},
keywordstyle=\color{keywords},
commentstyle=\color{comments},
stringstyle=\color{strings},
tabsize=2,
% Acentos, ñ, ¿, ¡ (tex.stackexchange.com/questions/24528)
extendedchars=true,
literate={á}{{\'a}}1 {é}{{\'e}}1 {í}{{\'i}}1 {ó}{{\'o}}1
         {ú}{{\'u}}1 {ñ}{{\~n}}1 {¡}{{\textexclamdown}}1
         {¿}{{?`}}1
}

% Gráficos
\usepackage{pgfplots}
\pgfplotsset{width=7cm,compat=1.8} % Opciones para gráficos

% Emoticonos
\usepackage{wasysym}

% tikz
\usepackage{tikz}
\usetikzlibrary{mindmap,trees,shadows}
\tikzset{ % Genera overlays
    invisible/.style={opacity=0},
    visible on/.style={alt={#1{}{invisible}}},
    alt/.code args={<#1>#2#3}{\alt<#1>{\pgfkeysalso{#2}}{\pgfkeysalso{#3}}},
}

%% Comandos %%
\newcommand{\ejemplo}[1]{\lstinputlisting{./examples/#1}} % Mostrar código de ejemplos
\newcommand{\muestra}[1]{\input{./examples/#1}}           % Mostrar ejemplos
\newcommand{\seccion}[1]{\input{./sections/#1}}           % Incluir secciones
\newcommand{\espacio}{\vspace*{\baselineskip}}            % Añade espacios
\newcommand{\beamer}{\texttt{beamer} }                    % Estilo único para beamer
\newcommand{\enlace}[3]{\href{#1}{\textbf{#2}} - {\small #3}}  % Estílo único para refs
\newcommand{\comando}[1]{{\color{black}\textbackslash}{\color{keywords}#1}}

%% Temas %%
% Tema y tema de color
  \usetheme{Szeged}
  \usecolortheme{crane}
% \useinnertheme{circles}
  \setbeamercovered{transparent}
% Colores bloques
%  \setbeamercolor{block title}{bg=title,fg=links}
%  \setbeamercolor{block body}{bg=backg,fg=black}
%  \setbeamercolor{block title alerted}{fg=red!70!black,bg=title!92!red}
%  \setbeamercolor{block body alerted}{fg=black,bg=backg}
%  \setbeamercolor{block title example}{fg=green!70!black,bg=title!92!green}
%  \setbeamercolor{block body example}{fg=black,bg=backg}
% Enlaces (tex.stackexchange.com/questions/13423)
\hypersetup{colorlinks,linkcolor=,urlcolor=links}
% Quita enlaces de navegación (stackoverflow.com/questions/3017030)
\setbeamertemplate{navigation symbols}{}
% Quita barra inferior (stackoverflow.com/questions/1435837)
\setbeamertemplate{footline}{}
% Evita warnings boxes
\hfuzz=20pt
\vfuzz=20pt
% Evita wranings itemize
\renewcommand\textbullet{\ensuremath{\bullet}}

% tikz
\usepackage{tikz}
\usetikzlibrary{shapes.multipart}

%% Título y otros %%
\title{Presentación práctica de eficiencia}                                               % Título
\subtitle{Asignatura: Algorítmica}                                  % Subtítulo
\author{Rubén Morales Pérez
		\and Francisco Javier Morales Piqueras
		\and Bruno Santindrian Manzanedo
		\and Ignacio de Loyola Barragan Lozano
		\and Francisco Leopoldo Gallego Salido}
\date{\today}                                                            % Fecha


%%%%%%%%%%%%%%%%%%%%%%%%%%%%%%%%%%%%%%%%%%%%%%%%%%%%%%%%%%%%%%%%

%% Presentación %%
\begin{document}

\begin{frame}
\titlepage
\end{frame}
\begin{frame}{Índice}
  \hypertarget{index}{}
  \tableofcontents
\end{frame}


\section{Presentación}
\subsection{Introducción}
\begin{frame}{Introducción}
	\begin{block}{Eficiencia}
	Divide y vencerás es una técnica algorítmica que consiste en resolver un problema 			diviéndolo en problemas más pequeños y combinando las soluciones. 
	El proceso de división continúa hasta que los subproblemas llegan a ser lo 					suficientemente sencillos como para una resolución directa.
	El hecho de que el tamaño de los subproblemas sea estrictamente menor que el tamaño 			original del problema nos garantiza la convergencia hacia los casos elementales.				\end{block}
\end{frame}

%%%%%%%%%%%%%%%%%%%%%%%%%%%%%%%%%%%%%%%%%%%%%%
\subsection{Automatización}
\begin{frame}{Scripts}
	\begin{block}{Script}
		Podemos obtener los datos fijando el n\'umero de vectores usados.
	\end{block}
	
	\begin{exampleblock}{script.sh}
	g++ -std=c++11 ../src/mezcla.cpp

	nelementos=10

	while [ \$nelementos -lt 2500 ]; do
    
    		./a.out \$nelementos 200 3
    
    		let nelementos=nelementos+25
	
	done
	\end{exampleblock}
\end{frame}

%%%%%%%%%%%%%%%%%%%%%%%%%%%%%%%%%%%%%%%%%%%%

\begin{frame}
	\begin{block}{Script}
		Si queremos fijar el n\'umero de vectores usaremos
	\end{block}
	
	\begin{exampleblock}{script.sh}
	kvectores=10

	while [ \$kvectores -lt 2500 ]; do
    
    		./a.out 200 \$kvectores 2
    	
    		let kvectores=kvectores+25

	done
	\end{exampleblock}
\end{frame}

%%%%%%%%%%%%%%%%%%%%%%%%%%%%%%%%%%%%%%%%%%%%

\begin{frame}
	\begin{block}{Script}
	Datos en 3 dimensiones, número de vectores, elementos del vector, y tiempo del algoritmo
	\end{block}
	
	\begin{exampleblock}{script.sh}
	nelementos=10

	nvectores=10

	while [ \$nelementos -lt 1000 ]; do
   	
   		./a.out \$nelementos 10 1
      		
   		...
   		
   		./a.out \$nelementos 910 1

   		let nelementos=nelementos+100
	done
	\end{exampleblock}
\end{frame}

%%%%%%%%%%%%%%%%%%%%%%%%%%%%%%%%%%%%%%%%%%%%

\begin{frame}{Scripts de gnuplot}
	\begin{block}{Gnuplot}		
		Los datos están en "datos.dat". Ejecutamos		
		\hspace{1cm}\$ gnuplot algoritmo.gp
	\end{block}
	\pause
	
	\begin{columns}

	\begin{column}{5cm}
	\begin{exampleblock}{algoritmo.gp}
	set terminal pngcairo
	
	set output "grafica.png"

	set title "..."

	set xlabel "Vectores/Elementos del vector"

	set ylabel "Tiempo (s)"

	set fit quiet

	f(x) = ...

	fit f(x) "datos.dat" via a

	plot "datos.dat", f(x)
	\end{exampleblock}
	\end{column}
	\pause
	
	\begin{column}{5cm}
	\begin{block}{Funciones ajustadas}
		\[f(x) = a*x\]
		\[g(x) = a*x*x\]
		\[h(x) = a*x*(log(x)/log(2))\]
	\end{block}
	\end{column}
	
	\end{columns}
\end{frame}

%%%%%%%%%%%%%%%%%%%%%%%%%%%%%%%%%%%%%%%
\subsection{Ordenador usado}
\begin{frame}{Ordenador usado}
	\begin{alertblock}{Ordenador usado para la ejecuci\'on}
	HP Pavilion g series (Pavilion g6)

	Sistema operativo: ubuntu 14.04 LTS

	Memoria: 3.8 GiB (4Gb)

	Procesador: Inter Core i3-2330M CPU @ 2.20GHz x 4

	Gráficos: Intel Sandybridge Mobile

	Tipo de SO: 64 bits

	Disco: 487.9 GB	
	\end{alertblock}
\end{frame}


%%%%%%%%%%%%%%%%%%%%%%%%%%%%%%%%%%%%%%

\section{Mezclando k vectores ordenados}
\begin{frame}{Problema}
	\begin{block}{Mezclando k vectores ordenados}
	Se tienen $k$ vectores ordenados (de menor a mayor), cada uno con $n$ elementos, y 				queremos combinarlos en un único vector ordenado (con $kn$ elementos)
	\end{block}

	\begin{block}{Cota superior}
	Es posible imponer una cota superior teórica. Teniendo en cuenta que hay kn elementos, 	si aplicásemos un algoritmo 	de ordenación con eficiencia $O(n) = nlog(n)$ deducimos 			que podemos encontrar un algoritmo de ordenación básica con eficiencia $O(k, n) = 
	nklog(nk)$. Tomar los $k$ vectores como uno solo no aprovecha aún el hecho de que 			partes del vector están ordenadas.
	\end{block}
\end{frame}

%%%%%%%%%%%%%%%%%%%%%%%%%%%%%%%%%%%%%%%
\subsection{Fuerza bruta}
\begin{frame}{Fuerza bruta}
	\begin{block}{Algoritmo}
	En cada paso elegimos el mínimo de los primeros elementos de los $k$ vectores, será el 	primer elemento del vector creciente resultante.
	Para el siguiente paso descartamos ese elemento y calculamos otra vez el mínimo, lo 			insertamos al final del vector resultante y así sucesivamente.

	Buscar el mínimo es $O(k)=k$ ya que el vector de índices tiene $k$ 							elementos, y lo repetimos $kn$ veces.
	\end{block}
	
	\begin{block}{Eficiencia}
	\[\sum_{i=1}^{kn}k = nk^2 \implies \ O(k,n)=nk^2\]
	\end{block}
\end{frame}

%%%%%%%%%%%%%%%%%%%%%%%%%%%%%%%%%%%%%%%%
\subsection{Divide y vencerás}
\begin{frame}{Divide y vencerás}
	\begin{block}{Algoritmo}
	Usaremos mergesort, pero con los primeros montículos ya creados, por tanto tendrá una 		constante oculta menor que usar mergesort para $kn$ datos arbitrarios. 
	
	En el proceso lo que haremos es ir mezclando las partes de dos en dos. El algoritmo 			que mezcla dos vectores en un único tiene eficiencia $O(n) = n$.
	\end{block}
\end{frame}

\begin{frame}
	\begin{block}{Código} % errorl
		\begin{figure}[h]
    		\centering
    		\includegraphics[width=0.9\textwidth]{./Imagenes/dyv.png}
    		\label{fig:mesh1}
		\end{figure}
	\end{block}
\end{frame}

%%%%%%%%%%%%%%%%%%%%%%%%%%%%%%%%%%%%%

\begin{frame}
	\begin{block}{Eficiencia}
	Donde $k$ es el n\'umero de vectores y $n$ el n\'umero de elementos de cada vector:s

	\[T(k,n) = \left \{ 
	\begin{matrix} 
		2n & 				\mbox{si } k=2
	\\ 2T(k/2,n) + kn & 		\mbox{si } k>2
	\end{matrix}
	\right.\]
	\end{block}
\end{frame}

\begin{frame}
	\begin{block}{Desarrollo}
	Sustituyendo $k=2^m \implies$ $T(2^m, n) = 2T(2^{m-1}, n) + 2^mn$
	\[T(2^m, n) = 2\left[ T(2^{m-2}, n) + 2^{m-1}n \right] + 2^mn\]
	\begin{center}
	Para el caso gen\'erico, con $j \in \left[0,m-1\right] \cap\mathbb{N}$ y 						desarrollando:
	\end{center}
	\[T(2^m, n)	= 2^jT(2^{m-j}, n) + \sum_{i=1}^{m-1} 2^mn\]
	\[T(2^m, n) = 2^{m-1} T(2, n) + \sum_{i=1}^{m-1} 2^mn\]
	\[T(2^m, n) = 2^mn + (m-1) 2^mn = 2^mn[1+(m-1)] = 2^mnm\]
	\end{block}
\end{frame}

\begin{frame}{Eficiencia final}
	\begin{block}{Solución}
	Deshacemos el cambio de variable, $k=2^m \implies log_2(k)=m$:
	\[T(k,n) = knlog_2k\]
	\end{block}
\end{frame}



\section{Eficiencie teórica fuerza bruta}
\section{Eficiencie teórica divide y vencerás}














%%%%%%%%%%%%%%%%%%%%%%%%%%%%%%%%%%%%%%%%%
\end{document}
