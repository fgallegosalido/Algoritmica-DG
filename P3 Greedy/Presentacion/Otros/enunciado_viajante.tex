En su formulación más sencilla, el problema del viajante de comercio (TSP, por Traveling
Salesman Problem) se define como sigue: dado un conjunto de ciudades y una matriz con
las distancias entre todas ellas, un viajante debe recorrer todas las ciudades exactamente una
vez, regresando al punto de partida, de forma tal que la distancia recorrida sea mínima. 
Mas formalmente, dado un grafo $G$, conexo y ponderado, se trata de hallar el ciclo hamiltoniano de
mínimo peso de ese grafo.

Una solución para TSP es una permutación del conjunto de ciudades que indica el orden
en que se deben recorrer. Para el cálculo de la longitud del ciclo no debemos olvidar sumar la
distancia que existe entre la última ciudad y la primera (hay que cerrar el ciclo).

Por su interés teórico y práctico, existe una variedad muy amplia de algoritmos para abordar
la solución del TSP y sus variantes (siendo un problema NP-Completo, el diseño y aplicación
de algoritmos exactos para su resolución no es factible en problemas de cierto tamaño). Nos
centraremos en una serie de algoritmos aproximados de tipo greedy y evaluaremos su rendimiento
en un conjunto de instancias del TSP. Para el diseño de estos algoritmos, utilizaremos
dos enfoques diferentes: a) estrategias basadas en alguna noción de cercanía, y b) estrategias
de inserción.

En el primer caso emplearemos la heurística del vecino más cercano, cuyo funcionamiento
es extremadamente simple: dada una ciudad inicial $v_0$, se agrega como ciudad siguiente aquella
$v_i$ (no incluída en el circuito) que se encuentre más cercana a $v_0$. El procedimiento se repite
hasta que todas las ciudades se hayan visitado.

En las estrategias de inserción, la idea es comenzar con un recorrido parcial, que incluya
algunas de las ciudades, y luego extender este recorrido insertando las ciudades restantes mediante
algún criterio de tipo greedy. Para poder implementar este tipo de estrategia, deben
definirse tres elementos:

1. Cómo se construye el recorrido parcial inicial.

2. Cuál es el nodo siguiente a insertar en el recorrido parcial.

3. Dónde se inserta el nodo seleccionado.

El recorrido inicial se puede construir a partir de las tres ciudades que formen un triángulo
lo más grande posible: por ejemplo, eligiendo la ciudad que está más al Este, la que está más al
Oeste, y la que está más al norte.

Cuando se haya seleccionado una ciudad, esta se ubicará en el punto del circuito que provoque
el menor incremento de su longitud total. Es decir, hemos que comprobar, para cada
posible posición, la longitud del circuito resultante y quedarnos con la mejor alternativa.

Por último, para decidir cuál es la ciudad que añadiremos a nuestro circuito, podemos
aplicar el siguiente criterio, denominado inserción más económica: de entre todas las ciudades
no visitadas, elegimos aquella que provoque el menor incremento en la longitud total del circuito.
En otras palabras, cada ciudad debemos insertarla en cada una de las soluciones posibles y
quedarnos con la ciudad (y posición) que nos permita obtener un circuito de menor longitud.
Seleccionaremos aquella ciudad que nos proporcione el mínimo de los mínimos calculados para
cada una de las ciudades.
